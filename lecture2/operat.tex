\section{Data Operators and Expressions}
\label{sec:oper}
\begin{frame}<beamer>
    \frametitle{Outline}
    \tableofcontents[currentsection]
\end{frame}

\begin{frame}[fragile]{Overview about Expressions}
\begin{itemize}
	\item {Legal expressions consist of legal combinations of}
	\begin{itemize}
		\item {Constants: const float PI = 3.14}
		\item {Variables: int a, b;}
		\item {Operators: +,-}
		\item {Function alls, printf("\%d", a)}
	\end{itemize}
\end{itemize}

\end{frame}

\begin{frame}[fragile]{Vadlid Operators in C}
\begin{itemize}
	\item {Operators}
	\begin{itemize}
		\item {Arithmetic: +,-,*, /, \%}
		\item {Relational: ==, !=, $>$,$<$, $<=$, $>=$}
		\item {Logical: \&\&, !, $||$}
		\item {\textcolor{yellow}{Bitwise}: \&, |, \^{ }, \~{ }}
		\item {\textcolor{yellow}{Shift}: $<<$, $>>$}
	\end{itemize}
\end{itemize}
\end{frame}

\begin{frame}[fragile]{Arithmetic Operators in C}
\begin{itemize}
	\item {Rules for operator precedence}
\end{itemize}
\begin{table}
\begin{center}
\begin{tabular}{|c|c|c|}
\hline 
Operator & Operation & Precedence \\ \hline \hline
() & Parenthese & Evaluated \textcolor{red}{first} \\ \hline
*,/ or \%&multiplication, division & evluated \textcolor{red}{second} \\ \hline
+ or - & addition, substraction & evaluated \textcolor{red}{last} \\ \hline
\hline
\end{tabular}
\end{center}
\end{table}
\begin{itemize}
	\item {Take average of three numbers}
	\item {1+2+4/3 ??}
\end{itemize}
\end{frame}

\begin{frame}[fragile]{Precedence Example}
\begin{equation}
(2+3+5)/3 \nonumber
\end{equation}
\begin{equation}
5*((2+6)%2) \nonumber
\end{equation}
\begin{columns}
\begin{column}{0.45\linewidth}
	\begin{lstlisting}[numbers=none, language=c]
	int avg = 2 + 3 + 5/3;
	float x=5*2+6%2;
	\end{lstlisting}
\end{column}
\begin{column}{0.45\linewidth}
	\begin{lstlisting}[numbers=none, language=c]
	int avg = (2 + 3 + 5)/3;
	float x=5*((2+6)%2);
	\end{lstlisting}
\end{column}
\end{columns}
\begin{itemize}
	\item {\textcolor{red}{Try to use ``()'' to clarify, if you are uncertain about the precedence}}
\end{itemize}
\end{frame}

\begin{frame}[fragile]{Division Operator (1)}
\begin{itemize}
	\item {Generates a result that is the same data type of \textcolor{red}{the largest operand} used in the operation}
	\item {Dividing two integers yields an integer result}
\end{itemize}
\begin{columns}
\begin{column}{0.4\linewidth}
	\begin{lstlisting}[numbers=none, language=c]
	5/2
	17/5
	\end{lstlisting}
\end{column}
\begin{column}{0.4\linewidth}
[Result]
	\begin{lstlisting}[numbers=none, language=c]
	2
	3
	\end{lstlisting}
\end{column}
\end{columns}
\end{frame}

\begin{frame}[fragile]{Division Operator (2)}
\begin{itemize}
	\item {Generates a result that is the same data type of \textcolor{red}{the largest operand} used in the operation}
	\item {Dividing two integers yields an integer result}
\end{itemize}
\begin{columns}
\begin{column}{0.4\linewidth}
	\begin{lstlisting}[numbers=none, language=c]
	5.0/2
	17.0/5
	\end{lstlisting}
\end{column}
\begin{column}{0.4\linewidth}
[Result]
	\begin{lstlisting}[numbers=none, language=c]
	2.5
	3.4
	\end{lstlisting}
\end{column}
\end{columns}
\end{frame}

\begin{frame}[fragile]{Modulus Operator \%}
\begin{itemize}
	\item {Modulus Operator \% returns the remainder}
	\item {Dividing two integers yields an integer result}
\end{itemize}
\begin{columns}
\begin{column}{0.4\linewidth}
	\begin{lstlisting}[numbers=none, language=c]
	5%2
	17%5
	12%3
	\end{lstlisting}
\end{column}
\begin{column}{0.4\linewidth}
[Result]
	\begin{lstlisting}[numbers=none, language=c]
	1
	2
	0
	\end{lstlisting}
\end{column}
\end{columns}
\end{frame}

\begin{frame}[fragile]{Evaluating Arithmetic Expressions (1)}
\begin{itemize}
	\item {See whether you can work out the answer}
\end{itemize}
\begin{columns}
\begin{column}{0.4\linewidth}
	\begin{lstlisting}[numbers=none, language=c]
	11/2
	11%2
	11/2.0
	5.0/2
	\end{lstlisting}
\end{column}
\begin{column}{0.4\linewidth}
[Result]
%	\begin{lstlisting}[numbers=none, language=c]
%
%	\end{lstlisting}
\end{column}
\end{columns}
\end{frame}


\begin{frame}[fragile]{Evaluating Arithmetic Expressions (2)}
\begin{itemize}
	\item {Check your answer}
\end{itemize}
\begin{columns}
\begin{column}{0.4\linewidth}
	\begin{lstlisting}[numbers=none, language=c]
	11/2
	11%2
	11/2.0
	5.0/2
	\end{lstlisting}
\end{column}
\begin{column}{0.4\linewidth}
[Result]
	\begin{lstlisting}[numbers=none, language=c]
	5
	1
	5.5
	2.5
	\end{lstlisting}
\end{column}
\end{columns}
\end{frame}

\begin{frame}[fragile]{Arithmetic Expressions (1)}
\begin{columns}
\begin{column}{0.4\linewidth}
[Arithmetic Expression]
\begin{equation}
	\begin{aligned}
	\frac{a}{b}\\
	2x \\
	\frac{x-7}{2+3y} \nonumber
	\end{aligned}
\end{equation}
\end{column}
\begin{column}{0.4\linewidth}
[Expression in C]
	\begin{lstlisting}[numbers=none, language=c]
	a/b
	2*x
	(x-7)/(2+3*y)
	\end{lstlisting}
\end{column}
\end{columns}
\end{frame}

\begin{frame}[fragile]{Arithmetic Expressions (2)}
\begin{columns}
\begin{column}{0.4\linewidth}
[Arithmetic Expression]
	\begin{lstlisting}[numbers=none, language=c, frame=none]
	2 * (-3)
	4 * 5 - 15
	4 + 2 * 5
	7/2
	7 / 2.0
	2 / 5
	2.0 / 5.0
	2 / 5 * 5
	2.0 + 1.0 + 5 / 2
	5 % 2
	4 * 5/2 + 5 % 2
	\end{lstlisting}
\end{column}
\begin{column}{0.4\linewidth}

\end{column}
\end{columns}
\end{frame}

\begin{frame}[fragile]{Arithmetic Expressions (3)}
\begin{columns}
\begin{column}{0.4\linewidth}
[Arithmetic Expression]
	\begin{lstlisting}[numbers=none, language=c, frame=none]
	2 * (-3)
	4 * 5 - 15
	4 + 2 * 5
	7/2
	7 / 2.0
	2 / 5
	2.0 / 5.0
	2 / 5 * 5
	2.0 + 1.0 + 5 / 2
	5 % 2
	4 * 5/2 + 5 % 2
	\end{lstlisting}
\end{column}
\begin{column}{0.4\linewidth}
[Results]
	\begin{lstlisting}[numbers=none, language=c, frame=none]
	-6
	5
	14
	3
	3.5
	0
	0.4
	0
	5.0
	1
	11
	\end{lstlisting}
\end{column}
\end{columns}
\end{frame}


\begin{frame}[fragile]{Data Assignment}
\begin{itemize}
	\item {Assig value to variable in accordance with its type}
\end{itemize}
\begin{columns}
\begin{column}{0.45\linewidth}
	\begin{lstlisting}[numbers=none, language=c, rulecolor=\color{blue}]
	int main()
	{ 
	   int a;
	   a = 2.99;
	   printf("a = %d", a);
	}
	\end{lstlisting}
\end{column}
\begin{column}{0.4\linewidth}
  [Output]
	\begin{lstlisting}[numbers=none, language=c, rulecolor=\color{blue}]
	a = 2
	\end{lstlisting}
\end{column}
\end{columns}
\begin{itemize}
	\item {Comments: above expression is valid, but \textcolor{red}{NOT} suggested}
\end{itemize}
\end{frame}

\begin{frame}[fragile]{Shortcut assignment Operators (1)}
\begin{table}
	\begin{center}
	\begin{tabular}{|l|l|} 
	\hline
	Assignment & Shortcut \\ \hline
	d = d - 4 & d-=4 \\ \hline
	e = e*5  & e *= 5 \\ \hline
	f = f/3 & f /= 3 \\ \hline
	g = g\%9 & g \%=9 \\ \hline \hline
	m = m*(5 + 3) & m *= 5+3 \\ \hline
	k = k/(5 + 1) & m /= 5+1 \\ \hline
	k = k/(5*7) & k /= 5*7 \\ \hline
	\end{tabular}
	\end{center}
\end{table}
\end{frame}

\begin{frame}[fragile]{Shortcut assignment Operators (2)}
\begin{columns}
\begin{column}{0.17\linewidth}
\end{column}
\begin{column}{0.65\linewidth}
	\begin{lstlisting}[numbers=none, language=c, rulecolor=\color{blue}]
		a += 4;		/* a = a + 4; */
		a -= 4;		/* a = a - 4; */
		a *= b;		/* a = a * b; */
		b /= 4+2;	/* b = b / (4+2); */
		b %= 2+3;	/* b = b % (2+3); */
	\end{lstlisting}
\end{column}
\begin{column}{0.17\linewidth}
\end{column}
\end{columns}
\end{frame}

\begin{frame}[fragile]{Shorthand Operators (1)}
\begin{itemize}
	\item {Incremental operator: \textbf{++}}
	\begin{itemize}
		\item {\textbf{i++} equivalent to \textbf{i = i+1}}
	\end{itemize}
	\item {Decremental operator: \textbf{--}}
	\begin{itemize}
		\item {\textbf{i--} equivalent to \textbf{i = i-1}}
	\end{itemize}
	\item {When they are used alone}
	\begin{itemize}
		\item {\textbf{i++} and \textbf{++i} behave the same as}
		\item {\textbf{i = i+1}}
		\item {Similar comment applies to \textbf{--}}
	\end{itemize}
\end{itemize}
\end{frame}

\begin{frame}[fragile]{Shorthand Operators (2)}
\begin{itemize}
	\item {When they appear in a compound expression, things are different}
	\item {\textbf{a=i++} will be different from \textbf{a=++i}}
	\item {In \textbf{a=i++}, \textbf{i} contributes its value to \textbf{a} first, then self-increments}
	\item {In \textbf{a=++i}, \textbf{i} self-increments first, then contributes its value to \textbf{a}}
	\item {Similiar comments apply to \textbf{i--} and \textbf{--i}}
\end{itemize}
\begin{columns}
\begin{column}{0.45\linewidth}
	\begin{lstlisting}[numbers=none, language=c, rulecolor=\color{blue}]
	int main()
	{
	   int a, b, i = 4;
	   a = i++;
	   b = ++i;
	}
	\end{lstlisting}
\end{column}
\begin{column}{0.45\linewidth}
	\begin{lstlisting}[numbers=none, language=c, rulecolor=\color{blue}]
	int main()
	{
	   int a, i = 4;
	   a = i;
	   i = i + 1;
	   i = i + 1;
	   b = i;
	}
	\end{lstlisting}
\end{column}
\end{columns}
\end{frame}

\begin{frame}[fragile]{Shorthand Operators (3)}
\begin{itemize}
	\item {Now verify how much you understand}
\end{itemize}
\begin{columns}
\begin{column}{0.65\linewidth}
	\begin{lstlisting}[numbers=none, language=c, rulecolor=\color{blue}]
	int main()
	{
	   int a, b, i = 4;
	   a = i--;
	   b = --i;
	   printf("a = %d, b = %d\n", a, b);
	}
	\end{lstlisting}
\end{column}
\begin{column}{0.3\linewidth}
[Output]
	\begin{lstlisting}[numbers=none, language=c, rulecolor=\color{blue}]
	a = ?, b = ?
	\end{lstlisting}
\end{column}
\end{columns}
\end{frame}

\begin{frame}[fragile]{Conditional Operator}
\begin{itemize}
	\item {Conditional Operator: logic\_exp1?exp2:exp3}
	\item {Three operands}
	\item {If logic\_exp1 is \textcolor{green}{none zero}, takes \textbf{exp2}}
	\item {If logic\_exp1 is \textcolor{green}{zero}, takes \textbf{exp3}}
\end{itemize}
\begin{columns}
\begin{column}{0.65\linewidth}
	\begin{lstlisting}[numbers=none, language=c, rulecolor=\color{blue}]
	int main()
	{
	   int a = 2, b = 3, i = 4;
	   a = b>i?b:i;
	   b = b==3?2:1;
	   printf("a = %d, b = %d\n", a, b);
	}
	\end{lstlisting}
\end{column}
\begin{column}{0.3\linewidth}
[Output]
	\begin{lstlisting}[numbers=none, language=c, rulecolor=\color{blue}]
	a = 4, b = 2
	\end{lstlisting}
\end{column}
\end{columns}
\end{frame}