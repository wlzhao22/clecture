\section{Pointer}
\label{sec:point}
\begin{frame}<beamer>
    \frametitle{Outline}
    \tableofcontents[currentsection]
\end{frame}

\begin{frame}
\frametitle{Pointer: flip (1)}
\begin{itemize}
	\item {Define a function to flip the elements of an array}
	\item {For instance:}
	a[7] =\{11, 4, 31, 2, 5, 12, 15\}
	\item {Change to}
	a[7] =\{15, 12, 5, 2, 31, 4, 11\}
\end{itemize}

\begin{itemize}
	\item {Requirements}
	\begin{enumerate}
	
		\item {Function looks like: \textcolor{blue}{void} \textbf{flip}(\textcolor{blue}{int} *a, \textcolor{blue}{int} sz)}
		\item {\textbf{*a} is the pointer pointing to array, \textbf{sz} is the length of array}
		\item {You should use pointer to visit the elements in the array}
		\item {Define a function \textcolor{blue}{void} \textbf{print}(\textcolor{blue}{int} *a, \textcolor{blue}{int} sz)}
		\item {Display the input array \textcolor{red}{before} and \textcolor{red}{after} you call \textbf{flip}}
	\end{enumerate}
\end{itemize}
\end{frame}


\begin{frame}
\frametitle{Pointer: flip (2)}
\vspace{-0.2in}
\begin{figure}
	\begin{center}
		\includegraphics[width=0.65\linewidth]{figs/flip_demo.pdf}
	\end{center}
\end{figure}
\end{frame}

\ifx\answers\undefined
\begin{frame}[fragile]
\frametitle{Pointer: flip (3)}
\vspace{-0.2in}
\begin{columns}
\begin{column}{0.460\linewidth}
\begin{lstlisting}[xleftmargin=0.03\linewidth]
#include <stdio.h>
void flip(int *a, int sz)}
{
  int *ps = a, i = 0, t;
  int *pe = a+sz-1;
  for(i = 0; i < sz/2; i++)
  {
    t   = *ps;
    *ps = *pe;
    *pe = t;
  }
}
\end{lstlisting}
\end{column}
\begin{column}{0.52\linewidth}
\begin{lstlisting}[xleftmargin=0.03\linewidth, firstnumber=13]
void print(int *a, int sz)
{
   int *p = a, i;
   for(i = 0; i < sz; i++, p++)
   {
     printf("%d ", *p);
     printf("\n");
   }
}
int main()
{
 int a[7] ={11,4,31,2,5,12,15};
 print(a, 7);
 flip(a,  7);
 print(a, 7);
 return 0;
}
\end{lstlisting}
\end{column}
\end{columns}
\end{frame}
\fi

\begin{frame}
\frametitle{Count frequency of each character in a string (1)}

\begin{itemize}
	\item {Given a string char str[] ="abcesZzmwrlmAnersfdasaf"}
	\item {Count the number of ocurrence of each alphabet}
	\item {Upper case and lower case are viewed as the same}
	\item {Output non-zero ocurrences}
	\item {Hints}
	\begin{itemize}
		\item {Use an integer array of 26 length to keep the counts}
		\item {Try to implement toLower(char str[]) by yourself}
	\end{itemize}
\end{itemize}
\end{frame}

\ifx\answers\undefined
\begin{frame}[fragile]
\frametitle{Count frequency of each character in a string (2)}
\vspace{-0.25in}
\begin{columns}
\begin{column}{0.46\linewidth}
\begin{lstlisting}[xleftmargin=0.05\linewidth]
#include <stdio.h>
#include <ctype.h>
int main()
{
  char str[]="abcEsZzmwr";
  int *p = str, i = 0;
  int counts[26] = {0};
  toLower(str);
  while(*p != '\0')
  {
    i = *p-'a';
    counts[i]=counts[i]+1;
    p++;
  }

\end{lstlisting}
\end{column}
\begin{column}{0.54\linewidth}
\begin{lstlisting}[xleftmargin=0.05\linewidth]
  for(i = 0; i < 26; i++)
  {
    if(counts[i])
    printf("%c: %d\n", 'a'+i, counts[i]);
  }
  return 0;
}
\end{lstlisting}
\end{column}
\end{columns}
\end{frame}
\fi

\ifx\answers\undefined
\begin{frame}[fragile]
\frametitle{Count frequency of each character in a string (3)}
\begin{lstlisting}[xleftmargin=0.1\linewidth, linewidth=0.8\linewidth]
#include <stdio.h>
#include <ctype.h>
void toLower(char str[])
{
   char *p = str;
   while(*p != '\0')
   {
	  *p = tolower(*p);
	   p++;
   }
}
\end{lstlisting}
\begin{itemize}
	\item {Put this function before ``main()''}
	\item {\textbf{tolower}(\textcolor{blue}{char} ch): convert one character to lower case}
\end{itemize}
\end{frame}
\fi

\begin{frame}
\frametitle{Count frequency of each alphabet in a string (4)}

\begin{itemize}
	\item {How about the code is now case-sensitive}
	\item {Count the number of ocurrence of each alphabet}
	\item {Output non-zero ocurrences}
	\item {Hints}
	\begin{itemize}
		\item {Use an 2D integer array of 26{$\times$}2 length to keep the counts}
		\item {int counts[26][2]}
	\end{itemize}
\end{itemize}
\end{frame}


\section{File}
\label{sec:file}
\begin{frame}<beamer>
    \frametitle{Outline}
    \tableofcontents[currentsection]
\end{frame}

\begin{frame}
\frametitle{File operation}
\begin{itemize}
	\item {Open a file "hi.txt"}
	\item {Write down "hello this is \textbf{your name}"}
	\item {Close the file}
	\item {Hints}
	\begin{itemize}
		\item {FILE *fp = fopen("C:/MyDocuments/hi.txt", "w");}
		\item {fprintf(fp, "hello this is xxx");}
		\item {fclose(fp);}
	\end{itemize}
\end{itemize}
\end{frame}

\ifx\answers\undefined
\begin{frame}[fragile]
\frametitle{File operation: open and write}
\begin{lstlisting}[xleftmargin=0.08\linewidth, linewidth=0.9\linewidth]
#include <stdio.h>
int main()
{
  char str[]="hello this is xxx";
  FILE *fp = fopen("C:/MyDocuments/hi.txt", "w");
  if(fp == NULL)
  {
    printf("File cannot open!\n");
    return 0;
  }
  fprintf(fp, str);
  fclose(fp);
}
\end{lstlisting}

\end{frame}
\fi

\ifx\answers\undefined
\begin{frame}[fragile]
\frametitle{File operation: open and read}
\begin{lstlisting}[xleftmargin=0.08\linewidth,linewidth=0.9\linewidth]
#include <stdio.h>
int main()
{
  char str[64]="";
  FILE *fp = fopen("C:/MyDocuments/hi.txt", "r");
  if(fp == NULL){
   printf("File cannot open!\n");
   return 0;
  }
  fscanf(fp, str);}
  fclose(fp);
  printf("%s\n", str);
}
\end{lstlisting}
\end{frame}
\fi

