\section{Exercises}
\label{sec:exec}
%\begin{frame}<beamer>
%    \frametitle{Outline}
%    \tableofcontents[currentsubsection]
%\end{frame}

\begin{frame}
\frametitle{Approximate exp(x) (1)}
\begin{equation}
	e^x=\sum_{n=0}^{\infty}\frac{x^n}{n!}=1+\frac{x}{1}+\frac{x^2}{2}+ \frac{x^3}{6}...
\end{equation}
\begin{itemize}
	\item {Requirements}
	\begin{itemize}
		\item {Keep terms that larger than 1e-6}
	\end{itemize}
	\item {Hints}
	\begin{itemize}
		\item {Define term=$\frac{x^n}{n!}$}
		\item {Do loop while \textcolor{blue}{abs}(term) is larger than 1e-6}
	\end{itemize}
\end{itemize}
\end{frame}

\ifx\answer\undefined
\begin{frame}[fragile]
\frametitle{Approximate exp(x) (2)}
\vspace{-0.15in}
\begin{lstlisting}[xleftmargin=0.08\linewidth, linewidth=0.9\linewidth]
#include <stdio.h>
#include <math.h>
int main()
{
  const double prec = 1e-6;
  double term = 1, sum = 0, x =0.3;
  double up = 1, low = 1;
  int i = 1;
  scanf("%lf", &x);
  while(abs(term) > prec)
  {
    sum += term;
     up  = up*x;
     low = low*i;
    term = up/low;
    i++;
  }
  printf("%lf\n", sum);
}
\end{lstlisting}
\end{frame}
\fi


\begin{frame}[fragile]
\frametitle{Print pyramid of alphabets on the screen}
\vspace{0.1in}
\begin{figure}
	\includegraphics[width=0.15\linewidth]{figs/stars.pdf}
\end{figure}
\begin{itemize}
	\item {Hints}
	\begin{itemize}
		\item {Suggested to use \textbf{for} loop}
		\item {Two levels of embeding}
	\end{itemize}
\end{itemize}
\begin{lstlisting}
#include <stdio.h>
for(i = 0; i < 5; i++)}
{
   for(j = 0; j < ?; j++)}
   {
     //filling your code   
   }
}
\end{lstlisting}
\end{frame}


\ifx\answer\undefined
\begin{frame}[fragile]
\frametitle{Answer}
\vspace{-0.15in}
\begin{columns}
\begin{column}{0.55\linewidth}
	\begin{lstlisting}[xleftmargin=0.02\linewidth]
#include <stdio.h>
int main()
{
  int i=0, j=0, count=0;
  int nl=5, nc=1, nb=nl-1;
  for(j = 0; j < nl; j++)
  {
    for(i = 0; i < nb; i++)
    {
       printf(" ");
    }
    nc = 2*j-1;
\end{lstlisting}
\end{column}
\begin{column}{0.45\linewidth}
\begin{lstlisting}[firstnumber=13]
   for(i=0;i<nc;i++)
   {
     printf("%c", ch);
   }
   ch++;
   nb--;
   printf("\n");
  }//for(j)
}
	\end{lstlisting}
\end{column}
\end{columns}
\end{frame}
\fi

\begin{frame}
\frametitle{Find out prime numbers}
\begin{itemize}
	\item {Requirements}
	\begin{itemize}
		\item {Find all prime numbers smaller than \textbf{500}}
		\item {Print out 8 numbers on each line}
	\end{itemize}
	\item{Hints}
	\begin{itemize}
		\item {Only dividable by itself}
		\item {By filtering method}
		\item {Try ``\%'' from \textbf{2} to \textbf{sqrt(num)}}
	\end{itemize}
\end{itemize}
\end{frame}

\ifx\answer\undefined
\begin{frame}[fragile]
\frametitle{Answer (1)}
\vspace{-0.15in}
\begin{lstlisting}[xleftmargin=0.05\linewidth, linewidth=0.9\linewidth]
#include <stdio.h>
#include <math.h>

int main()
{
   int _PRIME_ = 1;
   float b = 0;
   int i = 0, j = 0, count = 0;
   for(i = 2; i <= 500; i++)
   {
      b = sqrt(i+0.0);
      _PRIME_ = 1;
      for(j = 2; j < b; j++)
      {
          if(i%j == 0 && i != 2)
          {
             _PRIME_ = 0;
\end{lstlisting}

\end{frame}
\fi

\ifx\answer\undefined
\begin{frame}[fragile]
\frametitle{Answer (2)}
\vspace{-0.15in}
\begin{lstlisting}[firstnumber=18, xleftmargin=0.05\linewidth, linewidth=0.9\linewidth]
          }//if(i%j)
      }//for(j)
      if(_PRIME_ == 1)
      {
        count++;
        printf("%d\t", i);
        if(count%8 == 0)
        {
            printf("\n");
        }
      }//if
   }//for(i)
   if(count %8 !=0)
     printf("\n");
}
\end{lstlisting}

\end{frame}
\fi

\begin{frame}
\frametitle{Find out all complete numbers}
\begin{itemize}
	\item {Find out all the \textbf{complete number} in the range of [1, 10000]}
	\item {Complete number: it equals to the sum of its factors }
	\item {Example: 6 = 1+2+3}
\end{itemize}

\end{frame}

\ifx\answer\defined
\begin{frame}[fragile]
\frametitle{Answer}
\vspace{-0.15in}
\begin{columns}
\begin{column}{0.53\linewidth}
\begin{lstlisting}[xleftmargin=0.02\linewidth]
#include <stdio.h>
int main()
{
 int j=0, i=0, sum=0;
 for(j=1; j<=10000; j++)
 {
   sum = 1;
   for(i = 2; i < j; i++)
   {
     if(j%i == 0)
     {
       sum += i;
     }
   }//for(i)
}
\end{lstlisting}
\end{column}
\begin{column}{0.47\linewidth}
\begin{lstlisting}[firstnumber=16]
    if(sum == j)
    {
      printf("%d\t", j);
    }
  }//for(j)
  printf("\n");
}
\end{lstlisting}
\end{column}
\end{columns}

\end{frame}
\fi


\begin{frame}
\frametitle{Convert pure decimal fraction into binary form}
\begin{itemize}
	\item {Convert pure decimal fraction such as `0.635' into its binary form `0.1010001010001111010111'}
	\item {Accept a  pure decimal fraction from input: \textcolor{red}{0.625}}
	\item {Output its binary form: \textcolor{red}{0.101}}
	\item {The loop continues until the fraction is lower than \textcolor{red}{0.005}}
\end{itemize}

\end{frame}

\ifx\answer\defined
\begin{frame}[fragile]
\frametitle{Answer}
\vspace{-0.12in}
\begin{columns}
\begin{column}{0.53\linewidth}
\begin{lstlisting}[xleftmargin=0.02\linewidth]
#include <stdio.h>
int main()
{
    float a = 0.635;
    scanf("%f", &a);
    if(a < 1.0){
      printf("0.");
      do{
	      a = a*2;
          if(a >= 1.0)
    	  {
	         printf("1");
             a = a - 1.0;
          }else{
           	printf("0");
          }
      }while(a > 0.005);
\end{lstlisting}
\end{column}
\begin{column}{0.47\linewidth}
\begin{lstlisting}[firstnumber=18]
   }//end if
   return 0;
}
\end{lstlisting}
\end{column}
\end{columns}
\end{frame}
\fi