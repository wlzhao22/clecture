\section{Functions}
\label{sec:func}
\begin{frame}<beamer>
    \frametitle{Outline}
    \tableofcontents[currentsection]
\end{frame}

\begin{frame}{Maximum of three numbers (1)}

\begin{itemize}
	\item {The problem:}
	\begin{enumerate}
		\item {Given three float numbers input in ``main()''}
		\item {Work out a function to calculate their maximum}
		\item {\textcolor{blue}{int} max3(\textcolor{blue}{float} a, \textcolor{blue}{float} b, \textcolor{blue}{float} c)}
	\end{enumerate}
\end{itemize}

\end{frame}


\ifx\answer\undefined
\begin{frame}[fragile]{Maximum of three numbers (2)}
\vspace{-0.1in}
\begin{lstlisting}[xleftmargin=0.05\linewidth, linewidth=0.9\linewidth,basicstyle=\small]
#include <stdio.h>
float max3(float a, float b, float c)
{
    float maxm = a;
    if(maxm < b)
    {
        maxm = b;
    }
    if(maxm < c)
    {
        maxm = c;
    }
    return maxm;
}
void main()
{
    float a = 0, b = 0, c = 0, maxm = 0;
    scanf("%f%f%f", &a, &b, &c);
    maxm = max3(a, b, c);
    printf("result=%f", maxm);
}

\end{lstlisting}
\end{frame}
\fi

\begin{frame}{Greatest Common Divisor (1)}

\begin{itemize}
	\item {The problem:}
	\begin{enumerate}
		\item {Given two numbers input in ``main()''}
		\item {Work out a function to calculate their GCD}
		\item {\textcolor{blue}{int} gcd(\textcolor{blue}{int} a, \textcolor{blue}{int} b)}
	\end{enumerate}
\end{itemize}

\end{frame}

\ifx\answer\undefined
\begin{frame}[fragile]{Greatest Common Divisor (2)}
\begin{lstlisting}[xleftmargin=0.05\linewidth, linewidth=0.9\linewidth]
#include <stdio.h>
int gcd(int a, int b)
{
    int i = 2, g = 1;
    int m = a<b?a:b;
    for(i = 2; i <= m; i++)
    {
        if(a%i == 0 && b%i == 0)
        {
           g = i;
        }
    }
    return g;
}


\end{lstlisting}
\end{frame}
\fi

\ifx\answer\undefined
\begin{frame}[fragile]{Greatest Common Divisor (3)}
\begin{lstlisting}[firstnumber=19,xleftmargin=0.05\linewidth, linewidth=0.9\linewidth]
int main()
{
   int a = 0, b = 0, x;
   scanf("%d%d", &a, &b);
   x = gcd(a, b);
   printf("result=%d", x);
}
\end{lstlisting}
\end{frame}
\fi


\begin{frame}
\frametitle{Approximate cos(x) (1)}
\begin{equation}
	cos(x)=\sum_{n=1}^{\infty}\frac{(-1)^{(n-1)}x^{2n-2}}{(2n-2)!}=1-\frac{1}{2!}x^2+\frac{x^4}{4!}-\frac{x^6}{6!}+\frac{x^8}{8!}...
\end{equation}
\begin{itemize}
	\item {Requirements}
	\begin{itemize}
		\item {Define function \textbf{cosx(double x)}}
	\end{itemize}
\end{itemize}
\end{frame}

\ifx\answer\undefined
\begin{frame}[fragile]
\frametitle{Approximate cos(x) (2)}
\vspace{-0.27in}
\begin{columns}
\begin{column}{0.44\linewidth}
\begin{lstlisting}
#include <stdio.h>
double cosx(double x0)
{
  double t=1, sum=0, x=1;
  double fact = 1, pw = 1;
  int sign = 1, i = 0;
  x = x0;
  while(x >= 2*PI)
  {
    x = x- 2*PI;
  }
  while(x < 0)
  {
    x = x + 2*PI;
  }

  while(t > 0.0001)
  {
      sum += sign*t;
\end{lstlisting}
\end{column}
\begin{column}{0.53\linewidth}
\begin{lstlisting}[firstnumber=20]
      i   += 2;
      fact = i*(i-1)*fact;
      pw   = pw*x*x;
      sign = -1*sign;
      t    = pw/fact;
  }
  return sum;
}

int main()
{

  double val = 0, x = 0;

  scanf("%lf", &x);
  val = cosx(x);
  printf("result=%3.5lf", val);
  return 0;
}

\end{lstlisting}
\end{column}
\end{columns}
\end{frame}
\fi

%\begin{frame}{Count digit of an Integer (1)}
%
%\begin{itemize}
%	\item {The problem:}
%	\begin{enumerate}
%		\item {Given an input integer in ``main()''}
%		\item {Work out a function to count the digit for the input integer}
%		\item {For example, \textcolor{red}{n = 31}, \textcolor{red}{count}(n) returns \textcolor{red}{2}}
%		\item {and \textcolor{red}{n = -101}, \textcolor{red}{count}(n) returns \textcolor{red}{3}}
%		\item {print the digit in ``main()''}
%	\end{enumerate}
%\end{itemize}
%\begin{itemize}
%	\item {Requirements}
%	\begin{itemize}
%		\item {Define function \textcolor{red}{int count(int n)}}
%	\end{itemize}
%\end{itemize}
%\end{frame}
%
%\ifx\answer\undefined
%\begin{frame}[fragile]{Count digit for an Integer, the answer (1)}
%\vspace{-0.1in}
%\begin{lstlisting}[xleftmargin=0.05\linewidth, linewidth=0.9\linewidth]
%#include <stdio.h>
%int count(int a)
%{
%    int b = a;
%    int i = 0;
%    if(b < 0)
%       b = -b;
%
%    do
%    {
%        b = b/10;
%        i++;
%    }while(b > 0);
%    return i;
%}
%
%
%\end{lstlisting}
%\end{frame}
%\fi
%
%\ifx\answer\undefined
%\begin{frame}[fragile]{Count digit for an Integer: the answer (2)}
%\vspace{-0.1in}
%\begin{lstlisting}[xleftmargin=0.05\linewidth, linewidth=0.9\linewidth, firstnumber=23]
%int main()
%{
%    int a = 0, n = 0;
%    scanf("%d", &a);
%    n = count(a);
%    printf("result=%d", n);
%    return 0;
%}
%
%\end{lstlisting}
%\end{frame}
%\fi

\begin{frame}{Find the largest prime number (1)}

\begin{itemize}
	\item {The problem:}
	\begin{enumerate}
		\item {Given an input integer in ``main()''}
		\item {Find out a prime number that is the largest among all the prime numbers less than the given integer}
		\item {For example, \textcolor{red}{a = 31}, the largest prime number less than a is \textcolor{red}{29}}
		\item {print out the largest prime number in ``main()''}
	\end{enumerate}
\end{itemize}
\begin{itemize}
	\item {Requirements}
	\begin{itemize}
		\item {Define function \textcolor{red}{int isPrime(int a)}}
	\end{itemize}
\end{itemize}

\end{frame}

\ifx\answer\undefined
\begin{frame}[fragile]{Find the largest prime number (2)}
\begin{columns}
\begin{column}{0.44\linewidth}
\begin{lstlisting}[xleftmargin=0.03\linewidth]
#include <stdio.h>
int isPrime(int a)
{
   int i = 2;
   if(a < 2)
   {
           return 0;
   }else if(a == 2)
   {
           return 1;
   }
   for(i = 2; i < a; i++)
   {
       if(a%i == 0)
           return 0;
   }
   return 1;
}
\end{lstlisting}
\end{column}
\begin{column}{0.53\linewidth}
\begin{lstlisting}[firstnumber=15, linewidth=0.96\linewidth, xleftmargin=0.04\linewidth]
int main()
{
   int a = 0, i = 0;
   scanf("%d", &a);
   for(i = a-1; i > 1; i--)
   {
     if(isPrime(i))
     {
       printf("result=%d", i);
       break;
     }
   }
   return 0;
}
\end{lstlisting}
\end{column}
\end{columns}

\end{frame}
\fi

%
%\section{Variables}
%\label{sec:var}
%\begin{frame}<beamer>
%    \frametitle{Outline}
%    \tableofcontents[currentsection]
%\end{frame}
%
%
%\begin{frame}[fragile]
%\frametitle{\textbf{static}, \textbf{extern} and \textbf{auto} variables (1)}
%\vspace{-0.20in}
%\begin{columns}
%\begin{column}{0.43\linewidth}
%\begin{lstlisting}[xleftmargin=0.03\linewidth]
%#include <stdio.h>
%int a = 0;
%void count(int i);
%void multi();
%int main()
%{
% int i = 0;
% for(i = 1; i < 10; i++)
% {
%    count(i);
%    multi();
% }
% return 0;
%}
%\end{lstlisting}
%\end{column}
%\begin{column}{0.55\linewidth}
%\begin{lstlisting}[firstnumber=15, linewidth=0.96\linewidth, xleftmargin=0.04\linewidth]
%void count(int i)
%{
%  static a = 0;
%  a = a + 2;
%  printf("i=%d: a = %d\n",i,a);
%}
%
%void multi()
%{
%  a = 2*a;
%}
%\end{lstlisting}
%\end{column}
%\end{columns}
%
%\end{frame}


