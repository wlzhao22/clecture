\section{Exercises}
\label{sec:exec}
%\begin{frame}<beamer>
%    \frametitle{Outline}
%    \tableofcontents[currentsubsection]
%\end{frame}

\ifx\extra\undefined
\begin{frame}
\frametitle{Sort a randomly generated array: the problem}
%\vspace{-0.20in}
\begin{itemize}
	\item {Given an array sized of 20}
	\item {Sort the elements in ascending order: 1, 2, 4,$\cdots$}
	\item {General steps:}
\end{itemize}
\begin{enumerate}
	\item {Call rand()\%20 to generate an array}
	\item {Print the elements on the screen}
	\item {Repeat following loop 19 times}
	\item {~~For two neighboring elements do}
	\item {~~~~Switch their order if(a[j] $>$ a[j+1])}
	\item {~~End-for}
	\item {End-repeat}
	\item {Print the elements on the screen}
\end{enumerate}
\end{frame}

\ifx\answer\undefined
\begin{frame}[fragile]
\frametitle{Bubble Sort: the answer}
\vspace{-0.25in}
\begin{columns}
\begin{column}{0.53\linewidth}
\begin{lstlisting}[linewidth=0.95\linewidth, xleftmargin=0.05\linewidth]
#include <stdio.h>
#include <math.h>
#include <stdlib.h>
#define N 20
void bubsort()
{
 int i, j;
 float tmp, a[N] = {0};
 printf("Original:");
 for(i = 0; i < N; i++)
 {
   a[i] = rand()%20;
   printf("%3d ", a[i]);
 }
\end{lstlisting}
\end{column}
\begin{column}{0.47\linewidth}
\begin{lstlisting}[firstnumber=15]
 for(i = 0; i< N-1; i++)
 {
  for(j = 0; j < N-i-1; j++)
  {
    if(a[j] > a[j+1])
    {
      tmp    = a[j];
      a[j]   = a[j+1];
      a[j+1] = tmp;
    }
  }//for(j)
 }//for(i)
 printf("Sorted: ");
 for(i = 0; i < N; i++)
 {
   printf("%3d ", a[i]);
 }
}//end bubsort()
\end{lstlisting}
\end{column}
\end{columns}

\end{frame}
\fi
\fi

\begin{frame}
\frametitle{Count the frequency of chars in a string: the problem}
%\vspace{-0.15in}
\begin{itemize}
	\item {Given a string char s[] = "aa56231332bbAcc11"}
	\item {Count the frequency of each character in the string}
	\item {Output: a: -----$>$ 2}
	\item {~~~~~~~~~~~b: -----$>$ 2}
	\item {~~~~~~~~~~~c: -----$>$ 2}
	\item {~~~~~~~~~~~.....}
	\item {~~~~~~~~~~~1: -----$>$ 3}
\end{itemize}
\begin{itemize}
	\item {Define an array \textcolor{blue}{int} freq[128];}
	\item {If there is no occurrence for a character, \textcolor{red}{should not be displayed}}
\end{itemize}
\end{frame}


\ifx\answer\undefined

\begin{frame}[fragile]
\frametitle{Count the frequency of chars in a string: the answer}
\vspace{-0.2in}
\begin{lstlisting}[xleftmargin=0.05\linewidth, linewidth=0.90\linewidth]
#include <stdio.h>
#include <string.h>
int main()
{
   char s[] = "aa56231332bbAcc11";
   int  freq[128] ={0};
   int i = 0, len = strlen(s);
   for(i = 0; i < len; i++)
   {
      if(s[i]>=0 && s[i]<=127)
      {
         freq[s[i]] += 1;
      }
   }
   for(i = 0; i < 128; i++)
   {
       if(freq[i] > 0)
       {
          printf("%c ---> %d\n", i, freq[i]);
       }
   }
}//main()
\end{lstlisting}
\end{frame}
\fi

\begin{frame}
\frametitle{Insert numbers to a sorted array: the problem}
%\vspace{-0.15in}
\begin{itemize}
	\item {Given a sorted array $a[13]=\{10,14,15,18,19,21,22,31,32,35, 0, 0, 0\}$}
	\item {Insert elements of $b[3]=\{55, 17, 32\}$ one by one into \textbf{a}}
	\item {Keep the order of array \textbf{a} and print it out}
\end{itemize}

\end{frame}

\ifx\answer\undefined
\begin{frame}[fragile]
\frametitle{Insert numbers to a sorted array: the answer(1)}
\vspace{-0.2in}
\begin{lstlisting}[xleftmargin=0.04\linewidth, linewidth=0.96\linewidth]
#include <stdio.h>
int main()
{
   int a[13] = {10,14,15,18,19,21,22,31,32,35,0,0,0};
   int b[3] = {55, 17, 32};
   int i = 0, j = 0, c = 0;
   int len = 9, k = 0;
   for(i = 0; i < 3; i++)
   {
      for(j = 0; j <= len; j++)
      {
          if(b[i] < a[j])
            break;
      }
      for(k = len; k >= j; k--)
      {
           a[k+1] = a[k];
      }
      a[j] = b[i];
      len  = len + 1;
   }//for(i)
\end{lstlisting}
\end{frame}
\fi

\ifx\answer\undefined
\begin{frame}[fragile]
\frametitle{Insert numbers to a sorted array: the answer(2)}

\begin{lstlisting}[xleftmargin=0.05\linewidth, linewidth=0.9\linewidth,firstnumber=21]
   for(j = 0; j <= len; j++)
   {
       printf("%d ", a[j]);
   }
}//main()
\end{lstlisting}
\end{frame}
\fi


\begin{frame}[fragile]
\frametitle{Remove redundant blanks from a string: the problem}
%\vspace{-0.15in}
\begin{itemize}
	\item {Given a string str[]=``~~who~~am~~~~~i. To~be~~~~or~not~to~~be''}
	\item {Remove redundant blanks ` ' from this string}
	\item {The result should be ``who am i. To be or not to be''}
\end{itemize}
\begin{lstlisting}
void cleanStr(char str[])
{
   //filling your code
}

int main()
{
    char str[]="  who  am     i. To be    or not to  be    ";
    cleanStr(str);
    printf("%s\n", str);
}
\end{lstlisting}
\end{frame}

\ifx\answer\undefined
\begin{frame}[fragile]
\frametitle{Remove redundant blanks from a string: the answer (1)}
\vspace{-0.15in}
\begin{lstlisting}
void cleanStr(char str[])
{  int i = 0, j = 0, nblank = 0;
   char cstr[512];
   while(str[i] != '\0')
   {
       if(str[i] != ' '){
         cstr[j] = str[i];
         j++;
         nblank = 0;
       }else{
          nblank++;
          if(nblank <= 1){
            cstr[j] = str[i];
            j++;
          }
       }
       i++;
   }
   cstr[j] = '\0';
   strcpy(str, cstr);
}
\end{lstlisting}
\end{frame}
\fi

\ifx\answer\undefined
\begin{frame}[fragile]
\frametitle{Remove redundant blanks from a string: the answer (2)}
\vspace{-0.15in}
\begin{lstlisting}
void cleanStr(char str[])
{
   int i = 0, j = 0;
   int sz = strlen(str);
   while(i < sz)
   {
       while(i < sz && str[i] == ' ')
       {
          i++; 
       }
       for(; str[i] != ' ' && i < sz; i++, j++)
       {
           str[j] = str[i];
       }
       if(i < sz)
       {
          str[j++] = str[i++];
       }
   }//while(i)
   str[j] = '\0';
}
\end{lstlisting}
\end{frame}
\fi
