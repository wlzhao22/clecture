\section{Expressions}
\label{sec:exp}
\begin{frame}<beamer>
    \frametitle{Outline}
    \tableofcontents[currentsection]
\end{frame}

\begin{frame}[fragile]{Formatted Output (1)}
\begin{itemize}
	\item {Print out 6 numbers in two rows}
	\item {Three numbers in each row}
	\item {Each number occupies 5 digits and left-aligned}
\end{itemize}

\begin{lstlisting}[numbers=none, language=c, rulecolor=\color{blue}]
  123  451  332
  271  74   54
\end{lstlisting}
\end{frame}

\ifx\answer\undefined
\begin{frame}[fragile]{Formatted Output (2)}

\begin{lstlisting}[numbers=none, language=c, rulecolor=\color{blue}]
#include <stdio.h>
int main()
{
  printf("%-5d%-5d%-5d\n", 123, 451, 332);
  printf("%-5d%-5d%-5d", 271, 74, 54);
  return 0;
}
\end{lstlisting}

\end{frame}
\fi


\begin{frame}[fragile]{The octal and Hexadecimal form of a number (1)}
\begin{itemize}
	\item {Given an integer number, output its octal form}
	\item {Given an integer number, output its hexadecimal form}
\end{itemize}

\begin{lstlisting}[numbers=none, language=c, rulecolor=\color{blue}]
#include <stdio.h>
int main()
{
   int a = 0;
   scanf("%d", &a);
   return 0;
}
\end{lstlisting}
\end{frame}

\ifx\answer\undefined
\begin{frame}[fragile]{The octal and Hexadecimal form of a number (1)}
\begin{itemize}
	\item {Given an integer number, output its octal form}
	\item {Given an integer number, output its hexadecimal form}
\end{itemize}

\begin{lstlisting}[numbers=none, language=c, rulecolor=\color{blue}]
#include <stdio.h>
int main()
{
   int a = 0;
   scanf("%d", &a);
   printf("%o\n", a);
   printf("%x\n", a);
   return 0;
}
\end{lstlisting}
\end{frame}
\fi

\begin{frame}{Conditional Operator (1)}
\begin{itemize}
	\item {Problem:}
	\begin{itemize}
		\item {Accept two \textcolor{red}{integer} inputs \textbf{a} and \textbf{b} from user}
		\item {Output the square-root of the \textbf{maximum one}}
	\end{itemize}
	\item {Hints:}
	\begin{itemize}
		\item {Function for square-root is root=\textcolor{blue}{sqrt}(val)}
		\item {The header file is ``math.h''}
	\end{itemize}
\end{itemize}
\end{frame}

\ifx\answer\undefined
\begin{frame}[fragile]\frametitle{if-else clause}
\begin{lstlisting}[basicstyle=\large,xleftmargin=0.05\linewidth, linewidth=0.85\linewidth]
#include <stdio.h>
#include <math.h>
int main()
{
    int a = 0, b = 0, c = 0;
    float r = 0;
    printf("Input a:");
    scanf("%d", &a);
    printf("Input b:");
    scanf("%d", &b);
    c = a>b?a:b;
    r = sqrt(c);
    printf("Square-root is: %f\n", r);
    return 0;
}
\end{lstlisting}
\end{frame}
\fi


\begin{frame}
\frametitle{Take the average of several numbers (1)}
\begin{itemize}
	\item {Accept four real numbers from user, output their average}
\end{itemize}
\end{frame}

\ifx\answer\undefined
\begin{frame}[fragile]{Take the average of several numbers (2)}
\begin{lstlisting}[xleftmargin=0.05\linewidth, linewidth=0.85\linewidth]
#include <stdio.h>
int main( )
{
  float a = 0, b = 0;
  float c = 0, d = 0, s = 0;
  printf("Please input four numbers: ");
  scanf("%f%f%f%f", &a, &b, &c, &d);
  s = (a + b + c + d)/4.0;
  printf("The average is %f\n", s);
  return 0;
}
\end{lstlisting}
\end{frame}

\fi

\begin{frame}
\frametitle{Capitalize the character (1)}
\begin{itemize}
	\item {Accept the input from user a character}
	\item {Output a character in upper case if the input is in 'a'$\sim$'z' range}
	\item {Otherwise output the original character}
	\item {For example, input `a', output `A'}
	\item {~~~~input 'Z', output 'Z'; input ';', output ';'}
\end{itemize}

\end{frame}

\ifx\answer\undefined
\begin{frame}[fragile]{Capitalize the character (2)}
\begin{lstlisting}[xleftmargin=0.05\linewidth, linewidth=0.85\linewidth]
#include <stdio.h>
int main( )
{
  char ch = ' '; //assign it to blank
  printf("Please input a character: ");
  scanf("%c", &ch);
  ch = ch>='a' && ch <= 'z'?(ch - 32):ch;
  printf("The character is %c\n", ch);
  return 0;
}
\end{lstlisting}
\end{frame}
\fi

\begin{frame}[fragile]
\frametitle{Swap values of two variables (1)}
\begin{itemize}
	\item {Given variable a=31 and b = 17, swap the value of these two}
	\item {Output the values before and after the swapping}
	\item {You need another variable in order to fulfill the swapping}
\end{itemize}
\begin{lstlisting}[xleftmargin=0.05\linewidth, linewidth=0.85\linewidth]
a = 31, b = 17
a = 17, b = 31
\end{lstlisting}

\end{frame}

\ifx\answer\undefined
\begin{frame}[fragile]{Swap values of two variables (2)}
\begin{lstlisting}[xleftmargin=0.05\linewidth, linewidth=0.85\linewidth]
#include <stdio.h>
int main( )
{
  int a = 31, b = 17, c = 0;
  printf("a = %d, b = %d\n", a, b);
  c = a; 
  a = b;
  b = c;
  printf("a = %d, b = %d\n", a, b);
  return 0;
}
\end{lstlisting}
\end{frame}
\fi

