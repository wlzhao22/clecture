\section{Exercises}
\begin{frame}<beamer>
    \frametitle{Outline}
    \tableofcontents[currentsection]
\end{frame}
\label{sec:exec}

\begin{frame}\frametitle{if-else clause (1)}
\begin{itemize}
	\item {Read a character from input}
	\begin{enumerate}
		\item {If it is in 0$\sim$9, print out ``It is a digit''}
		\item {If it is in `a'$\sim$`z', convert it into upper case and print it out}
		\item {If it is in `A'$\sim$`Z', print it out directly}
		\item {If it is blank, print ``It is blank''}
		\item {Otherwise, print ``It is not a visible character''}
	\end{enumerate}
\end{itemize}
\end{frame}

\ifx\answer\defined
\begin{frame}[fragile]\frametitle{if-else clause (2)}
	\begin{lstlisting}[basicstyle=\large]
#include <stdio.h>
int main()
{
    char ch = ' ';
    ch = getchar();
    if(ch == ' ')
    {
        printf("It is blank\n");
    }else if(ch >= '0' && ch <= '9')
    {
        printf("It is digit\n");
    }
	\end{lstlisting}
\end{frame}

\begin{frame}[fragile]\frametitle{if-else clause (3)}
	\begin{lstlisting}[basicstyle=\large,firstnumber=12]
    else if(ch >= 'a' && ch <= 'z')
    {
        printf("%c", ch-32);
    }    
    else if(ch >= 'A' && ch <= 'Z')
    {
        printf("%c", ch);
    }else{
        printf("It is not visible character");
    }
    return 0;
}
	\end{lstlisting}
\end{frame}
\fi

%\begin{frame}
%\frametitle{if-else clause (1)}
%\vspace{-0.15in}
%\begin{itemize}
%	\item {Convert a score (0$\sim$100) to A - E levels}
%	\begin{enumerate}
%		\item {90 - 100: A}
%		\item {80 - 89: B}
%		\item {70 - 79: C}
%		\item {60 - 69: D}
%		\item {  $<$ 60: E}
%	\end{enumerate}
%	\item {The input should be a float}
%	\item {Print out the resulting level to the screen}
%\end{itemize}
%\begin{center}
%	81 -----$>$ B
%\end{center}
%
%\end{frame}
%
%\begin{frame}
%\frametitle{if-else clause (2)}
%\vspace{-0.15in}
%	\begin{enumerate}
%		\item {\#include $<$stdio.h$>$}
%		\item {int~main(~)}
%		\item {\{}
%		\item {~~~float score = 0;}
%   		\item {~~~char grade = '0';}
%		\item {~~~printf("Please input scores: ");}
%		\item {~~~scanf("\%f", \&score);}
%		\item {~~~if(score $<$ 0 $\parallel$ score $>$ 100)}
%		\item {~~~\{}
%   		\item {~~~~~printf("Input is invalid!${\setminus}$n");}
%		\item {~~~\}}
%		\item {~~~else\{}
%		\item {~~~~~if(score $>=$ 90)}
%		\item {~~~~~~~~grade = 'A';}
%		\item {~~~~~else if(score $>=$ 80)\{}
%		\item {~~~~~~~~grade = 'B';}
%		\item {~~~~~~\}else if(score $>=$ 70)\{}
%	\end{enumerate}
%\end{frame}
%
%\begin{frame}
%\frametitle{if-else clause (2)-continued}
%\begin{enumerate}
%	\setcounter{enumi}{17}
%		\item {~~~~~~~~~grade = 'C';}
%		\item {~~~~~~\}else if(score $>=$ 60)\{}
%		\item {~~~~~~~~~grade = 'D';}
%		\item {~~~~~~\}else \{}
%		\item {~~~~~~~~~~grade = 'E';}
%		\item {~~~~~~~\}}
%		\item {~~~\}//if-else}
%		\item {~~~printf("\%5.1f ---$>$ \%c${\setminus}$n", score, grade);}
%		\item {\}}
%\end{enumerate}
%\end{frame}

\begin{frame}
\frametitle{switch clause}
\begin{itemize}
	\item {Write C codes, which allows user to input number between 1 and 12 (the month)}
	\item {Then your code tells the user to which season the input month belongs}
\end{itemize}
\end{frame}

\ifx\answer\defined
\begin{frame}
	\begin{enumerate}
		\item {\#include $<$stdio.h$>$}
		\item {int~main(~)}
		\item {\{}
		\item {~~int m = 0;}
		\item {~~printf("Please input month: ");}
		\item {~~scanf("\%d", \&m);}
		\item {~if(m $<$ 1 $\parallel$ m $>$ 12)}
		\item {~\{}
		\item {~~~~printf("The input is invalid!$\setminus$n");}
		\item {~\} else\{}
		\item {~~switch(m)}
		\item {~~\{}
		\item {~~~case 1:}
		\item {~~~case 2:}
		\item {~~~case 3: printf("\%d --- Spring$\setminus$n", m); break;}
	\end{enumerate}
\end{frame}

\begin{frame}
	\begin{enumerate}
		\setcounter{enumi}{15}
		\item {~~~case 4:}
		\item {~~~case 5:}
		\item {~~~case 6:printf("\%d --- Summer$\setminus$n", m); break;}
		\item {~~~case 7:}
		\item {~~~case 8:}
		\item {~~~case 9:printf("\%d --- Autumn$\setminus$n", m); break;}
		\item {~~~case 10:}
		\item {~~~case 11:}
		\item {~~~case 12:printf("\%d --- Winter$\setminus$n", m); break;}
		\item {~~\}}
		\item {\}}
	\end{enumerate}
\end{frame}
\fi