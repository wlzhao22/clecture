\section{Editors and IDE}
\begin{frame}<beamer>
    \frametitle{Outline}
    \tableofcontents[currentsection]
\end{frame}

\begin{frame}{Programs for programming}
	\begin{itemize}
		\item {Integrated Development Environment}
		\begin{itemize}
			\item {VScode (\textcolor{blue}{default option})}
			\item {Dev-C++ (\textcolor{red}{recommended})}
			\item {Codeblocks}
		\end{itemize}
		\item {Any text editors + C compiler}
		\begin{itemize}
			\item {Microsoft C compiler (\textcolor{blue}{default option})}
			\item {GCC (\textcolor{red}{recommended})}
			\item {ICC: believed to be the most efficient one}
			\item {Turbo C: classic but being forgotten}
		\end{itemize}	
		\end{itemize}
\end{frame}

\section{Basic Ingradients of a C Program}
\label{sec:inter}
\begin{frame}<beamer>
    \frametitle{Outline}
    \tableofcontents[currentsection]
\end{frame}

\begin{frame}[fragile]{The first program}
	\begin{itemize}
		\item Create a new file named \textbf{main.c}.
		\item Open it in your text editor of choice.
		\item Fill it as follows:
	\end{itemize}
	\begin{lstlisting}[frame=non]
#include <stdio.h>
int main(void)
{
     printf("Hello World!\n");
     /* Print "Hello World!" on the command line */
     return 0;
}
	\end{lstlisting}
\end{frame}


\begin{frame}[fragile]{From source to bits}
	\centering
	Source code: main.c\\\
	$\Downarrow$\\\
\begin{columns}[T]
		\column{.40\textwidth}
Linux/Mac OS X
		\begin{lstlisting}[numbers=none]
gcc main.c -o hello
		\end{lstlisting}
		\column{.50\textwidth}
Windows
		\begin{lstlisting}[numbers=none]
D:\mycode\gcc main.c -o hello
		\end{lstlisting}
	\end{columns}
\vspace{-0.2in}
(Preprocessing $\rightarrow$ compiling $\rightarrow$ assembling $\rightarrow$ linking)
	\ \\\
	$\Downarrow$\\\ 
	Executable program\\\
	\begin{columns}[T]
		\column{.40\textwidth}
		Linux/Mac OS X (\textbf{\textcolor{green}{hello}})
		\begin{lstlisting}[numbers=none]
$ ./hello
$ Hello World!
		\end{lstlisting}
		\column{.50\textwidth}
		Windows (\textbf{hello.exe})
		\begin{lstlisting}[numbers=none]
D:\mycode\hello
Hello World!
D:\mycode\
		\end{lstlisting}
	\end{columns}
\end{frame}

\section{Program structure}
\label{sec:strt}
\begin{frame}<beamer>
    \frametitle{Outline}
    \tableofcontents[currentsection]
\end{frame}


\begin{frame}[fragile]{A basic program}
	\begin{columns}[T]
		\column{.6\textwidth}
		\begin{lstlisting}
#include <stdio.h>
int main()
{
   printf("Hello World!\n");
   /* Print "Hello World!" on the  command line */
   return 0;
}
		\end{lstlisting}
		\column{.4\textwidth}
		\ \\$\left. \begin{array}{c}\\\end{array}\right\rbrace $ Preprocessing statements
		\ \\\ \\$\left. \begin{array}{c}\\\\\\\\\\\\\end{array}\right\rbrace $ Main function
	\end{columns}
	\begin{itemize}
		\item {Processed before compilation}
		\item {Have their own language, start with a \textcolor{blue}{\#}}
		\item {In `stdio.h', function `\textbf{printf}()' has been defined}
	\end{itemize}
\end{frame}

\begin{frame}[fragile]{The main function}
	\begin{itemize}
		\item Basic function of every program
		\item Exists \textbf{exactly once} per program
		\item Called on program start
	\end{itemize}
	\begin{lstlisting}
	int main(void)
 	{
\end{lstlisting}
	\begin{itemize}
		\item As a function, \textit{main()} can take parameters and return a value
		\item Get used to \textit{void} and \textit{int}. They will be explained later
		\item '$\lbrace$' marks the start of the main function scope
	\end{itemize}
\end{frame}
\begin{frame}[fragile]{The main function scope}
	\begin{itemize}
		\item Contains program statements
		\item They are processed from top to bottom
	\end{itemize} \ \\
	\ \\
	\begin{lstlisting}
	return 0;
}
\end{lstlisting}
	\begin{itemize}
		\item Last statement, ends main function (and thus the whole program)
		\item \textit{0} tells the OS that everything went right
		\item '$\rbrace$' marks the end of the main function scope
	\end{itemize}
\end{frame}

\begin{frame}[fragile]{Statements}
	\begin{itemize}
		\item Instructions for the computer
		\item End with a \textit{;} (semicolon)
	\end{itemize}
	\begin{lstlisting}
	printf("Hello World!\n");
\end{lstlisting} \ \\ \ \\
	\begin{itemize}
		\item Here is the empty statement:
	\end{itemize}
	\begin{lstlisting}[numbers=none]
	;
\end{lstlisting}
	\begin{itemize}
		\item All statements are located in function blocks
	\end{itemize}
\end{frame}

\begin{frame}[fragile]{Comments}
	\begin{itemize}
		\item Information for the programmer, cut out before compilation
	\end{itemize}
	Single line comments:
	\begin{lstlisting}
	// Prints "Hello World!" on the command line
\end{lstlisting}
	Block comments (multi-line):
	\begin{lstlisting}
	/* Prints "Hello World!"
	   on the command line */
\end{lstlisting}
	Better style of block comments:
	\begin{lstlisting}
	/*
	 * Prints "Hello World!"
	 * on the command line
	 */
\end{lstlisting}
\end{frame}

\begin{frame}[fragile]{Order of execution}
	\begin{itemize}
		\item {Statements inside one function executed from top to bottom}
		\item {This is a convention for languages}
	\end{itemize}
\begin{columns}
\begin{column}{0.6\linewidth}
	\begin{lstlisting}[ xleftmargin=0.05\linewidth, linewidth=0.95\linewidth]
#include <stdio.h>
int main()
{
    printf("Hello China!\n");
    printf("Hello World!\n");
    printf("Hello Universe!\n");
    return 0;
}
\end{lstlisting}
\end{column}
\begin{column}{0.1\linewidth}
\end{column}
\begin{column}{0.3\linewidth}
	\begin{lstlisting}[frame=none, xleftmargin=0.01\linewidth]
Hello China!
Hello World!
Hello Universe!
	\end{lstlisting}
\end{column}
\end{columns}
\begin{itemize}
	\item {For \textcolor{red}{clarity}, \underline{one statement in one line}}
\end{itemize}

\end{frame}


