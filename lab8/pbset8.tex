\section{struct}
\label{sec:strt}
\begin{frame}<beamer>
    \frametitle{Outline}
    \tableofcontents[currentsection]
\end{frame}

\begin{frame}
\frametitle{struct (1)}
\begin{itemize}
	\item {Define a \textcolor{blue}{struct} for Complex number (\textbf{Compx})}
		\begin{itemize}
			\item {real (\textbf{real}) and virtual (\textbf{virt}) part}
		\end{itemize}
	\item {Define a function to perform multiplication between two complex numbers}
	\item {\textcolor{blue}{struct} Compx multComp(\textcolor{blue}{struct} Compx d1, \textcolor{blue}{struct} Compx d2)}
		\begin{itemize}
		    \item {....}
			\item {rslt.rl = a.rl*b.rl - a.vt*b.vt;}
			\item {rslt.vt = a.rl*b.vt + b.rl*a.vt;}
			\item {...}
		\end{itemize}
	\item {Define a function ``\textbf{void printCompx(\textcolor{blue}{struct} Compx a)}'' to print out a complex number}
	\begin{itemize}
		\item {It prints like following:\\}
		0.3+0.5i
	\end{itemize}
	
\end{itemize}

\end{frame}

\ifx\answers\undefined
\begin{frame}[fragile]
\frametitle{struct (2): the answer}
\vspace{-0.15in}
\begin{lstlisting}[xleftmargin=0.05\linewidth, linewidth=0.96\linewidth]
#include <stdio.h>
struct Compx{
  float rl;
  float vt;
};
struct Compx multComp(struct Compx d1, struct Compx d2)
{
	struct Compx r;
	r.rl = d1.rl*d2.rl-d1.vt*d2.vt;
	r.vt = d1.rl*d2.vt+d2.vt*d1.rl;
	return r;
}
void printComp(struct Compx r)
{
	if(r.vt > 0)
    	printf("%f+%fi\n", r.rl, r.vt);
    else if(r.vt < 0)
    	printf("%f%fi\n", r.rl, r.vt);
    else
        printf("%f\n", r.rl);
}
\end{lstlisting}
\end{frame}
\fi

\ifx\answers\undefined
\begin{frame}[fragile]
\frametitle{struct (3): the answer}
\begin{lstlisting}[firstnumber = 22, xleftmargin=0.05\linewidth, linewidth=0.96\linewidth]
int main()
{
	struct Compx d1 = {1.2,5.3}, d2 = {1.2,-1.3};
	struct Compx r = multComp(d1, d2);
	printComp(r);
	return 0;
}

\end{lstlisting}
\end{frame}
\fi


\begin{frame}
\frametitle{struct array (1)}
\begin{itemize}
	\item {Define a \textcolor{blue}{struct} named \textbf{NoteBook}}
	\begin{itemize}
		\item {qq: number, \textcolor{blue}{int} type}
		\item {name[32]: friend's name, \textcolor{blue}{char} type}
		\item {phone[16]: phone number, \textcolor{blue}{char} type}
	\end{itemize}
	\item {Define an array (\textbf{5 elements}) of NoteBook}
	\begin{itemize}
		\item {Input five records}
		\item {Output five records}
	\end{itemize}
	\item {Please check the size of your defined \textcolor{blue}{struct} type}
\end{itemize}

\end{frame}

\ifx\answers\undefined
\begin{frame}[fragile]
\frametitle{struct array (2)}
\vspace{-0.15in}
\begin{columns}
\begin{column}{0.38\linewidth}
\begin{lstlisting}[xleftmargin=0.04\linewidth]
#include <stdio.h>
struct NoteBook {
  long qq;
  char name[32];
  char phone[16];
};
typedef struct NoteBook QQBook;
void printQQbook();
int main()
{
  printQQbook();
  return 0;
}
\end{lstlisting}
\end{column}
\begin{column}{0.62\linewidth}
	\begin{lstlisting}
void printQQbook(){
{
  QQBook persons[4];
  int i = 0;
  for(i = 0; i < 4; i++)
  {
     printf("Name: ");
     scanf("%s", persons[i].name);
     printf("QQ: ");
     scanf("%d", &persons[i].qq);
  }
  for(i = 0; i < 4; i++){
  {
    printf("Name: %s\n", persons[i].name);
    printf("QQ: %d\n", persons[i].qq);
  }
}
\end{lstlisting}
\end{column}
\end{columns}
\end{frame}
\fi

\section{union}
\label{sec:union}
\begin{frame}<beamer>
    \frametitle{Outline}
    \tableofcontents[currentsection]
\end{frame}
\begin{frame}
\frametitle{union (1)}
\begin{itemize}
	\item {Define a \textcolor{blue}{union} type}
	\begin{itemize}
		\item {One \textbf{float} number}	
		\item {One \textbf{short} number}
		\item {One \textbf{char} character}
	\end{itemize}
	\begin{itemize}
		\item {Use \textcolor{blue}{typedef} to define type '\textbf{DATA}' of above union type}
		\item {Declare variable d1 of type \textbf{DATA}}
	\end{itemize}

\end{itemize}
\end{frame}

\ifx\answers\undefined
\begin{frame}[fragile]
\frametitle{union (2)}
\vspace{-0.15in}
\begin{columns}
\begin{column}{0.440\linewidth}
\begin{lstlisting}[xleftmargin=0.05\linewidth]
#include <stdio.h>
union Data {
  float f;
  char c;
  short i;
};
typedef union Data DATA;
void testUnion();
int main()
{
  testUnion();
  return 0;
}
\end{lstlisting}
\end{column}
\begin{column}{0.56\linewidth}
	\begin{lstlisting}[xleftmargin=0.05\linewidth]
void testUnion()
{
  DATA d1;
  printf("Size of data: %d\n", sizeof(DATA));
  d1.c = 'a';
  printf("%c\n", d1.c);
  d1.f = 3.1415;
  printf("f: %f\n", d1.f);
  printf("d: %d\n", d1.i);
  printf("c: %c\n", d1.c);
  d1.i = 9;
  printf("f: %f\n", d1.f);
  printf("i: %d\n", d1.i);
  printf("c: %c\n", d1.c);
 }
\end{lstlisting}
\end{column}
\end{columns}
\end{frame}
\fi


