\section{Overview about Computer Storage}
\label{sec:overview}
\begin{frame}<beamer>
    \frametitle{Outline}
    \tableofcontents[currentsection]
\end{frame}

\begin{frame}{Overview}
\begin{itemize}
	\item {When we doing programming}
	\item {Or run our code}
	\item {Two main components of a computer we work with}
	\begin{enumerate}
		\item {CPU: where our codes are executed}
		\item {Memory: where our codes are \textcolor{red}{temporarily} kept}
	\end{enumerate}
	\item {When computer is switched off}
	\item {Everything resides in memory disappear}
	\item {External storage is where we can keep things \textcolor{red}{permanently}}
\end{itemize}
\end{frame}

\begin{frame}{Storage Device (1)}
\begin{figure}
	\includegraphics[width=0.9\linewidth]{figs/storages.pdf}
\end{figure}
\begin{itemize}
	\item {There are all kinds of storage devices in the history of computer}
	\item {Currently, hard disc (it is hard), DVD and USB are popular}
	\item {Things are kept there in the unit of \textbf{file}}
\end{itemize}
\end{frame}

\begin{frame}{Storage Device (2)}
\begin{itemize}
	\item {There are several basic thing should be kept for a file}
	\begin{enumerate}
		\item {File name}
		\item {File size}
		\item {Location in the disc: which cylinder, which track and which sector?}
		\item {Time the file created, updated and the last time file is read}
		\item {Anything else??}
	\end{enumerate}
\end{itemize}
\begin{figure}
	\includegraphics[width=0.35\linewidth]{figs/disc.pdf}
\end{figure}
\end{frame}

\begin{frame}{Storage Device (3)}
\begin{itemize}
	\item {There are several basic thing should be kept for a file}
	\begin{enumerate}
		\item {File name}
		\item {File size}
		\item {Location in the disc: which cylinder, which track and which sector?}
		\item {Time the file created, updated and the last time file is read}
		\item {Logic location: \textcolor{red}{full path} of the file: ``c:/docs/codes/hi.c''}
	\end{enumerate}
	\item {We use a \textcolor{blue}{struct} type to define such kind of information}
\end{itemize}
\begin{figure}
	\includegraphics[width=0.5\linewidth]{figs/ftree.pdf}
\end{figure}
\end{frame}

\section{Operation on Files}
\label{sec:file}
\begin{frame}<beamer>
    \frametitle{Outline}
    \tableofcontents[currentsection]
\end{frame}

\begin{frame}{Flow of File Operation in C}
\begin{figure}
	\includegraphics[width=0.95\linewidth]{figs/flowfio.pdf}
\end{figure}
\begin{itemize}
	\item {When file is open (for write/read)}
	\item {A cache is created in the memory}
	\item {Data are put to cache from either side first}
	\item {Transfer to another side later}
\end{itemize}
\end{frame}

\begin{frame}[fragile]{Read File in C (1)}
%\vspace{-0.15in}
\begin{itemize}
	\item {A file name (with path) should be given}
	\begin{itemize}
		\item {full path: ``c:/docs/codes/hi.c''}
		\item {relative path: ``../codes/hi.c''}
	\end{itemize}
	\item {Tell the system, you are going to read ("r") the file}
\end{itemize}
\begin{lstlisting}[xleftmargin=0.06\linewidth, linewidth=0.85\linewidth]
#include <stdio.h>
int main()
{
  const char *fn = "c:/docs/codes/hi.txt";
  int a = 0, b = 0;
  FILE *fp = fopen(fn, "r");
  if(fp == NULL)//in case that open file failed
  { //required all the time
     printf("Open file %d\n failed", fn);
     return 0;
  }
  fscanf(fp, "%d%d", &a, &b);
  fclose(fp); //This is required all the time
  printf("%d %d\n", a, b);
  return 0;
}
\end{lstlisting}
\end{frame}


\begin{frame}[fragile]{FILE struct type}
\vspace{-0.15in}
\begin{columns}
\begin{column}{0.47\linewidth}
\begin{lstlisting}
#include <stdio.h>
int main()
{
  FILE *fp = fopen(fn, "r");
}
\end{lstlisting}
\end{column}
\begin{column}{0.49\linewidth}
\begin{lstlisting}[xleftmargin=0.1\linewidth, linewidth=0.90\linewidth]
typedef struct 
{
 short level;
 short token;
 short bsize;
 char fd;
 unsigned flags;
 unsigned char hold;
 unsigned char *buffer;
 unsigned char * curp;
 unsigned istemp; 
}FILE;
\end{lstlisting}
\end{column}
\end{columns}
\vspace{-0.2in}
\begin{itemize}
	\item {``FILE'' is a \textcolor{blue}{struct} designed for file operation}
	\item {It is defined in header ``$<$stdio.h$>$''}
\end{itemize}
\end{frame}


\begin{frame}[fragile]{Read File in C (2): the flow}
\tikzstyle{decision} = [diamond, draw, fill=yellow!20, 
    text width=6.5em, text badly centered, node distance=2cm, inner sep=0pt]
\tikzstyle{block} = [rectangle, draw, fill=blue!20, 
    text width=12em, text centered, rounded corners, minimum height=3em]
\tikzstyle{line} = [draw, -latex']
\tikzstyle{arrow} = [thick,->,>=stealth]
\tikzstyle{cloud} = [draw, ellipse,fill=red!20, node distance=2cm,
    minimum height=2em]
\begin{center}

\begin{tikzpicture}[node distance = 2.5cm, auto]
    % Place nodes
    \node [decision] (isopen) {if(fp != NULL)};
    \node [cloud, above of=isopen] (open) {FILE *fp = fopen("path", "r")};
    \node [block, below of=isopen, node distance=2.5cm] (read) {fscanf(fp, "\%s", oneline)};
    \node [cloud, below of=read, node distance=2cm] (close) {fclose(fp)};

    % Draw edges
    \path [line] (open) -- (isopen);
    %-- ++(3.0cm,0) |- node[anchor=west]
    \draw [arrow](isopen) -- ++(3.0cm,0) |- node[anchor=west]{no}(close);
    %\path [arrow] (isopen) -- node[near end]{no} (close);
    \path [line] (isopen) -- node [near end]{yes} (read);
    \path [line] (read) -- (close);
\end{tikzpicture}

\end{center}
\end{frame}

\begin{frame}[fragile]{Read File in C (3)}

\begin{lstlisting}[xleftmargin=0.06\linewidth, linewidth=0.85\linewidth]
#include <stdio.h>
int main()
{
  const char *fn = "c:/docs/codes/hi.txt";
  char line[1024];
  FILE *fp = fopen(fn, "r");
  if(fp == NULL){
     printf("Open file %d\n failed", fn);
     return ;
  }  
  fscanf(fp, "%s", line);
  fclose(fp); //This is required all the time
  printf("%s", line);
}
\end{lstlisting}
\vspace{-0.25in}
\begin{itemize}
	\item {It is possible that file does not exist}
	\item {Checking whether ``fp'' is NULL is safe to handle exceptions}
	\item {\textcolor{red}{Call ``fclose'' all the time when you finish reading}}
	\item {The \textbf{cache} will be always there}
\end{itemize}
\end{frame}

\begin{frame}[fragile]{Read File in C (1): the flow}
\tikzstyle{decision} = [diamond, draw, fill=yellow!20, 
    text width=6.5em, text badly centered, node distance=2cm, inner sep=0pt]
\tikzstyle{block} = [rectangle, draw, fill=blue!20, 
    text width=12em, text centered, rounded corners, minimum height=3em]
\tikzstyle{line} = [draw, -latex']
\tikzstyle{arrow} = [thick,->,>=stealth]
\tikzstyle{cloud} = [draw, ellipse,fill=red!20, node distance=2cm,
    minimum height=2em]
\begin{center}

\begin{tikzpicture}[node distance = 2.5cm, auto]
    % Place nodes
    \node [decision] (isopen) {if(fp != NULL)};
    \node [cloud, above of=isopen] (open) {FILE *fp = fopen("path", "w")};
    \node [block, below of=isopen, node distance=2.5cm] (read) {fprintf(fp, "hello world")};
    \node [cloud, below of=read, node distance=2cm] (close) {fclose(fp)};

    % Draw edges
    \path [line] (open) -- (isopen);
    %\path [line] (isopen) |- node [near end]{no} (close);
    \draw [arrow](isopen) -- ++(3.0cm,0) |- node[anchor=west]{no}(close);
    \path [line] (isopen) -- node [near end]{yes} (read);
    \path [line] (read) -- (close);
\end{tikzpicture}

\end{center}
\end{frame}

\begin{frame}[fragile]{Write File in C (2)}
%\vspace{-0.1in}
\begin{itemize}
	\item {A file name (with path) should be given}
	\begin{itemize}
		\item {full path: ``c:/docs/codes/hi.c''}
		\item {relative path: ``../codes/hi.c''}
	\end{itemize}
	\item {Tell the system, you are going to write ("w") the file}
\end{itemize}
\begin{lstlisting}[xleftmargin=0.06\linewidth, linewidth=0.85\linewidth]
#include <stdio.h>
int main()
{
  const char *fn = "c:/docs/codes/hi.txt";
  char line[1024];
  FILE *fp = fopen(fn, "w");
  if(fp == NULL)//in case that open file failed
  { //required all the time
     printf("Open file %d\n failed", fn);
     return 0;
  }
  fprintf(fp, "Hello world %d\n", 1);
  fclose(fp); 
  return 0;
}
\end{lstlisting}
\end{frame}

\begin{frame}[fragile]{Write File in C (3)}
\vspace{-0.15in}
\begin{lstlisting}[xleftmargin=0.06\linewidth, linewidth=0.85\linewidth]
#include <stdio.h>
int main()
{
  const char *fn = "c:/docs/codes/hi.txt";
  char line[1024];
  FILE *fp = fopen(fn, "w");
  if(fp == NULL){
     printf("Open file %d\n failed", fn);
     return ;
  }
  fprintf(fp, "Hello world %d\n", 1);
  fclose(fp); 
}
\end{lstlisting}
\vspace{-0.2in}
\begin{itemize}
	\item {It is possible that file does not exist}
	\item {Checking whether ``fp'' is NULL is safe to handle exceptions}
	\item {\textcolor{red}{Call ``fclose'' all the time when you finish writing}}
	\item {Otherwise no one can write/read the file}
\end{itemize}
\end{frame}

\begin{frame}[fragile]{Append File in C (1): the flow}
\tikzstyle{decision} = [diamond, draw, fill=yellow!20, 
    text width=6.5em, text badly centered, node distance=2cm, inner sep=0pt]
\tikzstyle{block} = [rectangle, draw, fill=blue!20, 
    text width=12em, text centered, rounded corners, minimum height=3em]
\tikzstyle{line} = [draw, -latex']
\tikzstyle{arrow} = [thick,->,>=stealth]
\tikzstyle{cloud} = [draw, ellipse,fill=red!20, node distance=2cm,
    minimum height=2em]
\begin{center}

\begin{tikzpicture}[node distance = 2.5cm, auto]
    % Place nodes
    \node [decision] (isopen) {if(fp != NULL)};
    \node [cloud, above of=isopen] (open) {FILE *fp = fopen("path", "a")};
    \node [block, below of=isopen, node distance=2.5cm] (read) {fprintf(fp, "hello world")};
    \node [cloud, below of=read, node distance=2cm] (close) {fclose(fp)};

    % Draw edges
    \path [line] (open) -- (isopen);
    %\path [line] (isopen) |- node [near end]{no} (close);
    \draw [arrow](isopen) -- ++(3.0cm,0) |- node[anchor=west]{no}(close);
    \path [line] (isopen) -- node [near end]{yes} (read);
    \path [line] (read) -- (close);
\end{tikzpicture}

\end{center}
\end{frame}

\begin{frame}[fragile]{Append File in C (2)}
\vspace{-0.15in}
\begin{itemize}
	\item {You allowed to put something more on the end of an existing file}
	\item {Tell the system, you are going to append ("a") the file}
\end{itemize}
\begin{lstlisting}[xleftmargin=0.06\linewidth, linewidth=0.85\linewidth]
#include <stdio.h>
int main()
{
  const char *fn = "c:/docs/codes/hi.txt";
  char line[1024];
  FILE *fp = fopen(fn, "a");
  if(fp == NULL)//in case that open file failed
  { //required all the time
     printf("Open file %d\n failed", fn);
     return 0;
  }
  fprintf(fp, "Hello world %d\n", 2);
  fclose(fp); //This is required all the time
  return 0;
}
\end{lstlisting}
\end{frame}

\begin{frame}[fragile]{Append File in C (3)}
\vspace{-0.15in}
\begin{lstlisting}[xleftmargin=0.06\linewidth, linewidth=0.85\linewidth]
#include <stdio.h>
int main()
{
  const char *fn = "c:/docs/codes/hi.txt";
  char line[1024];
  FILE *fp = fopen(fn, "a");
  if(fp == NULL)//in case that open file failed
  { //required all the time
     printf("Open file %d\n failed", fn);
     return 0;
  }
  fprintf(fp, "Hello world %d\n", 2);
  fclose(fp); //This is required all the time
  return 0;
}
\end{lstlisting}
\begin{itemize}
	\item {If the file does not exist}
	\item {It will be created}
\end{itemize}
\end{frame}

\begin{frame}[fragile]{Append File in C (3): the result}
\vspace{-0.15in}
\begin{lstlisting}
Hello world 1
Hello world 2
\end{lstlisting}
\end{frame}

\begin{frame}[fragile]{Summary on File Operation}
\begin{table}
	\begin{center}
		\begin{tabular}{|c|c|c|} \hline
		Operation & function & instruction \\ \hline
		Open & fopen(const char *fname, "r$|$w$|$a") &  \\ \hline
		Read & fscanf(FILE *fp, char *buffer) & "r" \\ \hline
		Write & fprintf(FILE *fp, "format string") & "w" \\ \hline
		Append & fprintf(FILE *fp, "format string") &  "a"\\ \hline
		Close & fclose(FILE *fp) &  \\ \hline
		\end{tabular}
	\end{center}
\end{table}
\begin{itemize}
	\item {Open and close operations are always required}
	\item {All above functions are defined in ``$<$stdio.h$>$''}
\end{itemize}
\end{frame}